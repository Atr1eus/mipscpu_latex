% !Mode:: "TeX:UTF-8"
%% 请使用 XeLaTeX 编译本文.
% \documentclass{WHUBachelor}% 选项 forprint: 交付打印时添加, 避免彩色链接字迹打印偏淡. 即使用下一行:
 \documentclass[forprint]{cpu}
%---------------------这里添加所需的package--------------------------------
\usepackage{url}

%--------------------------------------------------------------------------
\makeatletter
\def\BState{\State\hskip-\ALG@thistlm}
\makeatother
\begin{document}
%-----------------------------------------------------------------------------

%%%%%%% 下面的内容, 据实填空.

\Ccoursename{计算机设计实践} %课程名称
\title{计算机设计实践} %实验名称 换行请使用\\
\author{} % 学生姓名
\Csupervisor{\quad} %指导教师一姓名、职称
\CsupervisorAnother{无} %指导教师二姓名、职称
\CstudentNum{} %学号
\Cmajor{} % 专业名称
%\Cschoolname{国家网络安全学院} % 学院名
\date{二〇二X年X月} % 日期

%-----------------------------------------------------------------------------

\pdfbookmark[0]{封面}{title}         % 封面页加到 pdf 书签
\maketitle
\frontmatter
\pagenumbering{Roman}              % 正文之前的页码用大写罗马字母编号.2019.6.16:更新 正文之前的页码隐藏,无需显示
%-----------------------------------------------------------------------------
% !Mode:: "TeX:UTF-8"

%%% 此部分需要自行填写: 中文摘要及关键词 

%%% 郑重声明部分无需改动

%%%---- 郑重声明 (无需改动)------------------------------------%
\newpage
\thispagestyle{empty}
\vspace*{20pt}
\begin{center}{\ziju{0.8}\pmb{\songti\zihao{2} 郑重声明}}\end{center}
\par\vspace*{30pt}
\renewcommand{\baselinestretch}{2}

{\zihao{4}%

本人呈交的设计报告,是在指导老师的指导下,独立进行实验工作所取得的成果,
所有数据、图片资料真实可靠。 尽我所知,除文中已经注明引用的内容外,
本设计报告不包含他人享有著作权的内容。
对本设计报告做出贡献的其他个人和集体,
均已在文中以明确的方式标明。本设计报告的知识产权归属于培养单位。\\[2cm]

\hspace*{1cm}本人签名: $\underline{\hspace{3.5cm}}$
\hspace{2cm}日期: $\underline{\hspace{3.5cm}}$\hfill\par}
%------------------------------------------------------------------------------
\baselineskip=23pt  % 正文行距为 23 磅
%------------------------------------------------------------------------------





%%======摘要===========================%
\begin{cnabstract}
\thispagestyle{empty}

XXXX实验的实验目的是XXXX。\\
实验设计主要遵循XXXX。\\
实验内容主要包括:\\
实验结论为XXXX   \\



\end{cnabstract}
\par
\vspace*{2em}


%%%%--  关键词 -----------------------------------------%%%%%%%%
%%%%-- 注意: 每个关键词之间用“;”分开,最后一个关键词不打标点符号
\cnkeywords{关键词1;关键词2;关键词3;}



    % 加入摘要, 申明.
%==========================把目录加入到书签==============================%%%%%%


\tableofcontents
\thispagestyle{empty}				%不显示罗马数字 ——zmx更新于2019.06.18
\addtocontents{toc}{\protect\thispagestyle{empty}}




\mainmatter %% 以下是正文
%%%%%%%%%%%%%%%%%%%%%%%%%%%--------main matter-------%%%%%%%%%%%%%%%%%%%%%%%%%%%%%%%%%%%%
\pagestyle{plain}%plain
%\cfoot{\thepage{\zihao{5}\bf\usefonttimes}}
%\renewcommand{\baselinestretch}{1.6}
%\setlength{\baselineskip}{23pt}
\baselineskip=23pt  % 正文行距为 23 磅

%此处书写正文-------------------------------------------------------------------------------------


\chapter{实验目的和意义}
 
 \section{实验目的}

 \begin{description}

\item 本实验。。。。。。


\end{description}
%pdf 文件也可以反向搜索! \CJKunderwave{双击~pdf 中要修改的文字, 将直接跳转到源文件中相应位置.}
%
\chapter{实验环境介绍}

\section{Verilog HDL}
\section{MARS}
\section{ModelSim}
\section{Vivado}
\section{Nexys4DDR}


\chapter{概要设计}
\section{总体设计}
\begin{description}
    \item (介绍你的CPU的总体情况,单周期还是多周期,多周期的状态机,支持的指令等)
\end{description}


\section{PC(程序计数器)(仿照PPT里的写法)}
\subsection{功能描述}
\subsection{模块接口}
\begin{table}[!ht]
    \centering
    \begin{tabular}{|l|l|l|}
    \hline
        信号名 &  方向 & 描述 \\ \hline
        ~ & ~ & ~ \\ \hline
        ~ & ~ & ~ \\ \hline
        ~ & ~ & ~ \\ \hline
        ~ & ~ & ~ \\ \hline
        ~ & ~ & ~ \\ \hline
        ~ & ~ & ~ \\ \hline
        ~ & ~ & ~ \\ \hline
        ~ & ~ & ~ \\ \hline

    \end{tabular}
\end{table}
\section{RF(寄存器文件)}
\subsection{功能描述}
\subsection{模块接口}


\chapter{详细设计}
\section{CPU总体结构}
    (画出你的CPU总体结构图)
\section{PC(程序计数器)(放代码和描述说明)}

\section{RF(寄存器文件)}

\chapter{测试结果及分析}
\section{仿真代码及分析}
\section{仿真测试结果}
(展示指令执行结果)
\section{下载测试代码及分析}
\section{下载测试结果}

\chapter{实验心得}
(实验中遇到的问题,以及如何解决问题。)


% !Mode:: "TeX:UTF-8"
%%%%%%%%%%%%%%%%%%%%%%%%%%%%-------结论--------%%%%%%%%%%%%%%%%%%%%%%%%%%%%%%%%

\acknowledgement
\addcontentsline{toc}{chapter}{结论}
%\linespread{1.5}

这里写本次实验的结论。

% 这里写本次实验的结论
















 %%%结论


%%%=== 参考文献 ========%%%
\cleardoublepage\phantomsection
\addcontentsline{toc}{chapter}{参考文献}
\renewcommand{\baselinestretch}{1.6}
\begin{thebibliography}{00}


  \bibitem{r1} 李亚民.计算机原理与设计——Verilog HDL版[B].清华大学出版社,2011年6月



\end{thebibliography}



%%%-------------- 附录. 不需要可以删除.-----------


\appendix
%%%-------------- 教师评语评分 ------------------
\begin{teacher}
\thispagestyle{empty}
评语: 
\par
\vspace*{12.5cm}
\hspace*{7.5cm}评分: 
\vspace*{1cm}

\hspace*{7.3cm}评阅人:

\vspace*{0.5cm}

\hspace*{10.1cm}年\hspace*{1cm}月\hspace*{1cm}日

\vspace*{0.5cm}

{\songti \zihao{4} \makebox[1cm][s]{(备注:对该实验报告给予优点和不足的评价,并给出百分制评分。)}}

\end{teacher}


\cleardoublepage
\end{document}





